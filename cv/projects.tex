%!TEX root = ./cv.tex
\textbf{Current Projects}
\begin{enumerate}
\setcounter{enumi}{0}
\item \textbf{Cray OpenSHMEMX}
    \begin{itemize}
        \item Cray OpenSHMEMX, a new OpenSHMEM library implementation that
        supersedes the existing production ready Cray SHMEM library on future
        exascale Cray Shasta architectures
        \item Design and develop Cray OpenSHMEMX from scratch with support for
        different transport layers including DMAPP, XPMEM, and Libfabric
    \end{itemize}

\item \textbf{Cray MPICH}
    \begin{itemize}
        \item Create an optimized message matching layer for the MPI
        implementation
        \item Optimize GPU-awareness in the MPI implementation
    \end{itemize}
\end{enumerate}

\textbf{Past Projects}
\begin{enumerate}
\setcounter{enumi}{0}
\item \textbf{Cray SHMEM}\\
    \textbf{Project description:} Cray SHMEM is a production quality SHMEM
    implementation on different Cray platforms with OpenSHMEM standards
    compliant. \textbf{My Contributions:} I am particularly involved in
    maintaining library along with implementing new features as per the
    OpenSHMEM standards, also propose and prototype new features like
    Communication Contexts, Symmetric Memory Partitions and Teams(PE-Subsets).

\item \textbf{Cray Global Arrays(Cray-GA) ComEx-DMAPP}\\
    \textbf{Project description:} Global Arrays is a PGAS library from Pacific
    Northwest National Laboratory (PNNL). \textbf{My Contributions:} I was
    involved in optimizing and maintain the ComEx - DMAPP communication layer.

\item \textbf{OpenSHMEM Reference Implementation}\\
    \textbf{Project description:} University of Houston worked with \textit{Oak
    Ridge National Laboratory} to standardize OpenSHMEM via a community-driven
    specification with a reference implementation. \textbf{My Contributions:}
    I have been particularly involved in the optimization of the collective
    communication performance.

\item \textbf{Coarray Fortran}\\
    \textbf{Project description:} Coarray Fortran is a set of new language
    features incorporated into the Fortran 2008 standard to enable parallel
    programming in Fortran with minimal changes to the language syntax. It is a
    joint project between University of Houston and \textit{Total}
    \textbf{My Contributions:} I was particularly involved in the optimization
    of the Coarray Fortran runtime library with new underlying communication
    layers.
\end{enumerate}
