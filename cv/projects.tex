\textbf{Current Projects}
\begin{enumerate}
\setcounter{enumi}{0}
\item \textbf{Cray SHMEM}
    \begin{itemize}
        \item optimize and maintain Cray SHMEM for different platforms
        \item implement new features as per the OpenSHMEM standards
        \item propose and prototype new features like Communication
        Contexts, Symmetric Memory Partitions and Teams(PE-Subsets)
    \end{itemize}

\item \textbf{Cray MPICH}
    \begin{itemize}
        \item create an optimized message matching layer for the MPI
        implementation.
    \end{itemize}

\item \textbf{Cray Global Arrays(Cray-GA)}
    \begin{itemize}
        \item optimize and maintain the ComEx - DMAPP communication layer
    \end{itemize}
\end{enumerate}

\textbf{Past Projects}
\begin{enumerate}
\setcounter{enumi}{0}
\item \textbf{OpenSHMEM Reference Implementation}\\
      \textbf{Project description:} University of Houston worked with \textit{Oak
      Ridge National Laboratory} to standardize OpenSHMEM via a community-driven
      specification with a reference implementation. \textbf{My contributions:}
      I have been particularly involved in the optimization of the collective
      communication performance.

\item \textbf{Coarray Fortran}\\
     \textbf{Project description:} Coarray Fortran is a set of new language
     features incorporated into the Fortran 2008 standard which enable parallel
     programming in Fortran with minimal changes to the language syntax. It is a
     joint project between University of Houston and \textit{Total}
     \textbf{My contributions:} I am particularly involved in the optimization
     of the Coarray Fortran runtime library with new underlying communication
     layers.
\end{enumerate}
